\documentclass[12pt]{article} % размер шрифта

\usepackage{tikz} % картинки в tikz
\usepackage{microtype} % свешивание пунктуации

\usepackage{array} % для столбцов фиксированной ширины

\usepackage{url} % для вставки ссылок \url{...}

\usepackage{indentfirst} % отступ в первом параграфе

\usepackage{sectsty} % для центрирования названий частей
\allsectionsfont{\centering} % приказываем центрировать все sections

\usepackage{amsthm} % теоремы и доказательства

\theoremstyle{definition} % прямой шрифт в условии теорем
\newtheorem{theorem}{Теорема}[section]


\usepackage{amsmath, amssymb} % куча стандартных математических плюшек

\usepackage[top=2cm, left=1.5cm, right=1.5cm, bottom=2cm]{geometry} % размер текста на странице

\usepackage{lastpage} % чтобы узнать номер последней страницы

\usepackage{enumitem} % дополнительные плюшки для списков
%  например \begin{enumerate}[resume] позволяет продолжить нумерацию в новом списке
\usepackage{caption} % подписи к картинкам без плавающего окружения figure


\usepackage{fancyhdr} % весёлые колонтитулы
\pagestyle{fancy}
\lhead{Машинка}
\chead{}
\rhead{2018-11-23, Тестик-2}
\lfoot{}
\cfoot{}
\rfoot{\thepage/\pageref{LastPage}}
\renewcommand{\headrulewidth}{0.4pt}
\renewcommand{\footrulewidth}{0.4pt}



\usepackage{todonotes} % для вставки в документ заметок о том, что осталось сделать
% \todo{Здесь надо коэффициенты исправить}
% \missingfigure{Здесь будет картина Последний день Помпеи}
% команда \listoftodos — печатает все поставленные \todo'шки

\usepackage{booktabs} % красивые таблицы
% заповеди из документации:
% 1. Не используйте вертикальные линии
% 2. Не используйте двойные линии
% 3. Единицы измерения помещайте в шапку таблицы
% 4. Не сокращайте .1 вместо 0.1
% 5. Повторяющееся значение повторяйте, а не говорите "то же"

\usepackage{fontspec} % поддержка разных шрифтов
\usepackage{polyglossia} % поддержка разных языков

\setmainlanguage{russian}
\setotherlanguages{english}

\setmainfont{Linux Libertine O} % выбираем шрифт
% если Linux Libertine не установлен, то
% можно также попробовать Helvetica, Arial, Cambria и т.Д.

% чтобы использовать шрифт Linux Libertine на личном компе,
% его надо предварительно скачать по ссылке
% http://www.linuxlibertine.org/index.php?id=91&L=1

% на сервисах типа sharelatex.com этот шрифт есть :)

\newfontfamily{\cyrillicfonttt}{Linux Libertine O}
% пояснение зачем нужно шаманство с \newfontfamily
% http://tex.stackexchange.com/questions/91507/

\AddEnumerateCounter{\asbuk}{\russian@alph}{щ} % для списков с русскими буквами
\setlist[enumerate, 2]{label=\asbuk\cdot),ref=\asbuk\cdot} % списки уровня 2 будут буквами а) б) ...

%% эконометрические и вероятностные сокращения
\DeclareMathOperator{\Cov}{Cov}
\DeclareMathOperator{\Corr}{Corr}
\DeclareMathOperator{\Var}{Var}
\DeclareMathOperator{\E}{E}
\def \hb{\hat{\beta}}
\def \hs{\hat{\sigma}}
\def \htheta{\hat{\theta}}
\def \s{\sigma}
\def \hy{\hat{y}}
\def \hY{\hat{Y}}
\def \v1{\vec{1}}
\def \e{\varepsilon}
\def \he{\hat{\e}}
\def \z{z}
\def \hVar{\widehat{\Var}}
\def \hCorr{\widehat{\Corr}}
\def \hCov{\widehat{\Cov}}
\def \cN{\mathcal{N}}


\begin{document}


\textbf{Время на выполнение: 20 минут}

\begin{enumerate}

\item (2 балла) Количество мёда в горшках, которое Винни-Пух хочет съесть в гостях, можно представить в виде $y_i = 5 + u_i$, где $u_i$ имеет равномерное распределение $u_i \sim \mathcal{U}[-2;2]$, а $i$ — номер визита в гости.

При прогнозировании аппетита Винни-Пуха Кролик абсолютно игнорирует все его прошлые визиты и просто подкидывает правильную монетку восемь раз. При каждом выпадении орла, Кролик ставит на стол очередной горшок мёда из глубоких запасов.

Постройте разложение квадратичной функции потерь прогноза Кролика на компоненты дисперсии, смещения и шума.

%Для квадратичной функции потерь, ошибки можно разложить на следующие компоненты:
% $$L = \mathbb{E}_{x}\left[\left(\mu(X) - \mathbb{E_X}[\mu(X)]\right)^2\right] + \mathbb{E}_{x,y}\left[\left(y - \mathbb{E}[y|x]\right)^2\right] + \mathbb{E}_{x}\left[\left(\mathbb{E}_{X}[\mu(X)] - \mathbb{E}[y|x]\right)^2\right] $$

% объясните, что означает каждое слагаемое (warning: слагаемые не в классическом порядке).

% \item (1 балл) Как связана сложность модели и смещение? Как связана сложность модели и разброс? Какую проблему решает бустинг? Какие деревья лучше использовать в алгоритме случайного леса?

\item (1 балл) Герман делит дата сайнтистов на три типа: специалисты по deep learning (20\%), machine learning (60\%), big data (20\%). Посчитайте индекс Джини и энтропию.

\item (1 балл) Постройте пример, в котором индекс Джини больше $0.99$.

% \item (2 балла) Кот машин-лёрнера Василия поймал три рыбки весом 300, 600, 1200 граммов. Найдите закон распределения бутстреп статистики для разницы между самой легкой и самой тяжелой рыбой.

\item (2 балла) Постройте регрессионное дерево для прогнозирования $y$ с помощью $x$ на обучающей выборке. Узлы делятся до тех пор, пока в узле остается больше двух наблюдений.

\begin{tabular}{rrrrrrrr}
\toprule
$y$ & 100 & 102 & 103 & 50  & 55 & 61 & 70  \\
$x$ & 1 & 2 & 3 & 4  & 5 & 6 & 7  \\
\bottomrule
\end{tabular}

\end{enumerate}




\end{document}
